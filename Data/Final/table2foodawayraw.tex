\begin{table}[tbp] \centering
\newcolumntype{C}{>{\centering\arraybackslash}X}

\caption{Real food away from home spending before and after retirement by wealth tertiles (PSID category).}
\begin{tabularx}{\textwidth}{lCCCC}

\toprule
{Wealth\_Tertiles}&{First}&{Second}&{Third}&{All} \tabularnewline
\midrule\addlinespace[1.5ex]
Means:&&&& \tabularnewline
\midrule Pre-retirement&897&1,654&2,246&1,597 \tabularnewline
Post-retirement&799&1,632&2,391&1,624 \tabularnewline
Percent Change in Means&-11.0&-2.5&16.2&1.7 \tabularnewline
\midrule Medians:&&&& \tabularnewline
\midrule Pre-retirement&393&1,043&1,706&983 \tabularnewline
Post-retirement&360&932&1,608&932 \tabularnewline
Percent Change in Medians&-8.3&-10.7&-5.7&-5.2 \tabularnewline
Median Percent Change (p10)*&-100.0&-81.2&-69.8&-87.2 \tabularnewline
Median Percent Change (p25)*&-77.4&-52.2&-38.4&-52.9 \tabularnewline
Median Percent Change (p50)&-18.8&-5.8&-3.4&-5.8 \tabularnewline
Median Percent Change (p75)*&88.5&63.8&61.1&65.8 \tabularnewline
Median Percent Change (p90)*&318.8&352.3&186.2&262.1 \tabularnewline
\bottomrule \addlinespace[1.5ex]

\end{tabularx}
\begin{flushleft}
\footnotesize *These values are not medians but percentiles, as indicated in the parentheses. \linebreak --- \linebreak This table references Table 2 of Hurd and Rohwedder's paper: Heterogeneity in spending change at retirement. \linebreak --- \linebreak This spending category is defined by dining out in CAMS. \linebreak --- \linebreak Mean percent change is not reported because observation error on spending can produce large outliers when spending is put in ratio form. \linebreak --- \linebreak N = 893.
\end{flushleft}
\end{table}
