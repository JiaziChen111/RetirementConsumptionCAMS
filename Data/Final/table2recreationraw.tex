\begin{table}[tbp] \centering
\newcolumntype{C}{>{\centering\arraybackslash}X}

\caption{Real recreation spending before and after retirement by wealth tertiles (PSID category).}
\begin{tabularx}{\textwidth}{lCCCC}

\toprule
{Wealth\_Tertiles}&{First}&{Second}&{Third}&{All} \tabularnewline
\midrule\addlinespace[1.5ex]
Means:&&&& \tabularnewline
\midrule Pre-retirement&966&2,253&4,011&2,374 \tabularnewline
Post-retirement&641&1,752&4,457&2,329 \tabularnewline
Percent Change in Means&-33.7&-51.9&46.2&-1.9 \tabularnewline
\midrule Medians:&&&& \tabularnewline
\midrule Pre-retirement&238&1,115&2,385&1,035 \tabularnewline
Post-retirement&164&1,052&2,859&1,009 \tabularnewline
Percent Change in Medians&-31.4&-5.7&19.9&-2.5 \tabularnewline
Median Percent Change (p10)*&-1.0&-1.0&-0.7&-1.0 \tabularnewline
Median Percent Change (p25)*&-0.9&-0.7&-0.4&-0.6 \tabularnewline
Median Percent Change (p50)&-0.3&-0.1&0.1&-0.1 \tabularnewline
Median Percent Change (p75)*&0.9&1.0&1.0&1.0 \tabularnewline
Median Percent Change (p90)*&3.6&3.4&3.1&3.4 \tabularnewline
\bottomrule \addlinespace[1.5ex]

\end{tabularx}
\begin{flushleft}
\footnotesize *These values are not medians but percentiles, as indicated in the parentheses. \linebreak --- \linebreak This table references Table 2 of Hurd and Rohwedder's paper: Heterogeneity in spending change at retirement. \linebreak --- \linebreak This spending category is defined by vacations, tickets, hobbies/sports, hobbies, and sports in CAMS. \linebreak --- \linebreak Mean percent change is not reported because observation error on spending can produce large outliers when spending is put in ratio form. \linebreak --- \linebreak N = 919.
\end{flushleft}
\end{table}
