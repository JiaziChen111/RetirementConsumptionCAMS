\begin{table}[tbp] \centering
\newcolumntype{C}{>{\centering\arraybackslash}X}

\caption{Real recreation spending before and after retirement by social security income tertiles (PSID category).}
\begin{tabularx}{\textwidth}{lCCCC}

\toprule
{SS\_Tertiles}&{First}&{Second}&{Third}&{All} \tabularnewline
\midrule\addlinespace[1.5ex]
Means:&&&& \tabularnewline
\midrule Pre-retirement&1,052&2,015&3,042&2,374 \tabularnewline
Post-retirement&1,422&1,637&3,298&2,329 \tabularnewline
Percent Change in Means&35.1&-35.9&24.3&-1.9 \tabularnewline
\midrule Medians:&&&& \tabularnewline
\midrule Pre-retirement&245&624&1,686&1,035 \tabularnewline
Post-retirement&515&616&1,931&1,009 \tabularnewline
Percent Change in Medians&109.7&-1.1&14.5&-2.5 \tabularnewline
Median Percent Change (p10)*&-94.3&-100.0&-86.8&-98.0 \tabularnewline
Median Percent Change (p25)*&-53.3&-71.7&-54.4&-63.2 \tabularnewline
Median Percent Change (p50)&30.0&-16.8&1.0&-5.5 \tabularnewline
Median Percent Change (p75)*&178.4&92.4&92.4&95.3 \tabularnewline
Median Percent Change (p90)*&415.7&328.3&339.8&335.1 \tabularnewline
\bottomrule \addlinespace[1.5ex]

\end{tabularx}
\begin{flushleft}
\footnotesize *These values are not medians but percentiles, as indicated in the parentheses. \linebreak --- \linebreak This table references Table 2 of Hurd and Rohwedder's paper: Heterogeneity in spending change at retirement. \linebreak --- \linebreak This spending category is defined by vacations, tickets, hobbies/sports, hobbies, and sports in CAMS. \linebreak --- \linebreak Mean percent change is not reported because observation error on spending can produce large outliers when spending is put in ratio form. \linebreak --- \linebreak N = 919.
\end{flushleft}
\end{table}
