\begin{table}[tbp] \centering
\newcolumntype{C}{>{\centering\arraybackslash}X}

\caption{Real food at home spending before and after retirement by social security income tertiles (PSID category).}
\begin{tabularx}{\textwidth}{lCCCC}

\toprule
{SS\_Tertiles}&{First}&{Second}&{Third}&{All} \tabularnewline
\midrule\addlinespace[1.5ex]
Means:&&&& \tabularnewline
\midrule Pre-retirement&3,929&3,607&4,344&3,952 \tabularnewline
Post-retirement&3,794&3,722&4,297&3,977 \tabularnewline
Percent Change in Means&-3.4&2.9&-1.2&0.6 \tabularnewline
\midrule Medians:&&&& \tabularnewline
\midrule Pre-retirement&3,386&3,000&3,774&3,354 \tabularnewline
Post-retirement&2,844&2,928&3,625&3,195 \tabularnewline
Percent Change in Medians&-16.0&-2.4&-3.9&-4.7 \tabularnewline
Median Percent Change (p10)*&-72.4&-72.2&-64.6&-69.5 \tabularnewline
Median Percent Change (p25)*&-48.5&-40.3&-33.5&-35.7 \tabularnewline
Median Percent Change (p50)&-11.7&-3.4&-1.4&-3.4 \tabularnewline
Median Percent Change (p75)*&67.2&58.9&44.6&52.8 \tabularnewline
Median Percent Change (p90)*&233.9&172.7&157.7&173.0 \tabularnewline
\bottomrule \addlinespace[1.5ex]

\end{tabularx}
\begin{flushleft}
\footnotesize *These values are not medians but percentiles, as indicated in the parentheses. \linebreak --- \linebreak This table references Table 2 of Hurd and Rohwedder's paper: Heterogeneity in spending change at retirement. \linebreak --- \linebreak This spending category is defined by food/drink in CAMS. \linebreak --- \linebreak Mean percent change is not reported because observation error on spending can produce large outliers when spending is put in ratio form. \linebreak --- \linebreak N = 896.
\end{flushleft}
\end{table}
