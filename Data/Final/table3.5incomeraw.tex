\begin{table}[tbp] \centering
\newcolumntype{C}{>{\centering\arraybackslash}X}

\caption{Real household income before and after retirement by social security income tertiles.}
\begin{tabularx}{\textwidth}{lCCCC}

\toprule
{SS\_Tertiles}&{First}&{Second}&{Third}&{All} \tabularnewline
\midrule\addlinespace[1.5ex]
Means:&&&& \tabularnewline
\midrule Pre-retirement&40,449&56,167&87,404&68,208 \tabularnewline
Post-retirement&39,575&49,130&94,130&67,573 \tabularnewline
Percent Change in Means&-2.2&-17.4&16.6&-0.9 \tabularnewline
\midrule Medians:&&&& \tabularnewline
\midrule Pre-retirement&24,848&36,364&65,091&48,920 \tabularnewline
Post-retirement&21,184&33,840&63,720&46,512 \tabularnewline
Percent Change in Medians&-14.7&-6.9&-2.1&-4.9 \tabularnewline
Median Percent Change (p10)*&-0.9&-0.7&-0.5&-0.6 \tabularnewline
Median Percent Change (p25)*&-0.5&-0.3&-0.2&-0.3 \tabularnewline
Median Percent Change (p50)&-0.1&0.0&0.0&0.0 \tabularnewline
Median Percent Change (p75)*&0.2&0.3&0.3&0.3 \tabularnewline
Median Percent Change (p90)*&1.5&1.1&0.8&1.0 \tabularnewline
\bottomrule \addlinespace[1.5ex]

\end{tabularx}
\begin{flushleft}
\footnotesize *These values are not medians but percentiles, as indicated in the parentheses. \linebreak --- \linebreak This table references Table 3 of Hurd and Rohwedder's paper: Heterogeneity in spending change at retirement. \linebreak --- \linebreak Mean percent change is not reported because observation error on spending can produce large outliers when spending is put in ratio form. \linebreak --- \linebreak N = 919.
\end{flushleft}
\end{table}
