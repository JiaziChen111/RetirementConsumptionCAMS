\begin{table}[tbp] \centering
\newcolumntype{C}{>{\centering\arraybackslash}X}

\caption{Real food spending before and after retirement by social security income tertiles (Generated category).}
\begin{tabularx}{\textwidth}{lCCCC}

\toprule
{SS\_Tertiles}&{First}&{Second}&{Third}&{All} \tabularnewline
\midrule\addlinespace[1.5ex]
Means:&&&& \tabularnewline
\midrule Pre-retirement&4,609&4,878&6,292&5,460 \tabularnewline
Post-retirement&4,556&4,927&6,342&5,501 \tabularnewline
Percent Change in Means&-1.2&1.1&1.1&0.8 \tabularnewline
\midrule Medians:&&&& \tabularnewline
\midrule Pre-retirement&4,064&3,963&5,266&4,523 \tabularnewline
Post-retirement&3,354&3,888&5,145&4,288 \tabularnewline
Percent Change in Medians&-17.5&-1.9&-2.3&-5.2 \tabularnewline
Median Percent Change (p10)*&-0.6&-0.7&-0.6&-0.7 \tabularnewline
Median Percent Change (p25)*&-0.5&-0.4&-0.3&-0.4 \tabularnewline
Median Percent Change (p50)&0.0&0.0&0.0&0.0 \tabularnewline
Median Percent Change (p75)*&0.6&0.5&0.4&0.5 \tabularnewline
Median Percent Change (p90)*&2.0&1.8&1.4&1.7 \tabularnewline
\bottomrule \addlinespace[1.5ex]

\end{tabularx}
\begin{flushleft}
\footnotesize *These values are not medians but percentiles, as indicated in the parentheses. \linebreak --- \linebreak This table references Table 2 of Hurd and Rohwedder's paper: Heterogeneity in spending change at retirement. \linebreak --- \linebreak This spending category is defined by food/drink and dining out in CAMS. \linebreak --- \linebreak Mean percent change is not reported because observation error on spending can produce large outliers when spending is put in ratio form. \linebreak --- \linebreak N = 919.
\end{flushleft}
\end{table}
