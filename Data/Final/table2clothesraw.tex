\begin{table}[tbp] \centering
\newcolumntype{C}{>{\centering\arraybackslash}X}

\caption{Real clothing spending before and after retirement by wealth tertiles (PSID category).}
\begin{tabularx}{\textwidth}{lCCCC}

\toprule
{Wealth\_Tertiles}&{First}&{Second}&{Third}&{All} \tabularnewline
\midrule\addlinespace[1.5ex]
Means:&&&& \tabularnewline
\midrule Pre-retirement&788&985&1,176&979 \tabularnewline
Post-retirement&606&688&1,291&872 \tabularnewline
Percent Change in Means&-23.1&-37.7&14.6&-10.9 \tabularnewline
\midrule Medians:&&&& \tabularnewline
\midrule Pre-retirement&283&469&624&471 \tabularnewline
Post-retirement&213&400&509&392 \tabularnewline
Percent Change in Medians&-24.5&-14.8&-18.4&-16.9 \tabularnewline
Median Percent Change (p10)*&-1.0&-0.9&-0.8&-0.9 \tabularnewline
Median Percent Change (p25)*&-0.8&-0.6&-0.6&-0.7 \tabularnewline
Median Percent Change (p50)&-0.3&-0.2&-0.2&-0.2 \tabularnewline
Median Percent Change (p75)*&0.9&0.5&0.3&0.5 \tabularnewline
Median Percent Change (p90)*&2.4&2.1&1.5&1.9 \tabularnewline
\bottomrule \addlinespace[1.5ex]

\end{tabularx}
\begin{flushleft}
\footnotesize *These values are not medians but percentiles, as indicated in the parentheses. \linebreak --- \linebreak This table references Table 2 of Hurd and Rohwedder's paper: Heterogeneity in spending change at retirement. \linebreak --- \linebreak This spending category is defined by clothing in CAMS. \linebreak --- \linebreak Mean percent change is not reported because observation error on spending can produce large outliers when spending is put in ratio form. \linebreak --- \linebreak N = 882.
\end{flushleft}
\end{table}
