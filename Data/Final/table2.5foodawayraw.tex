\begin{table}[tbp] \centering
\newcolumntype{C}{>{\centering\arraybackslash}X}

\caption{Real food away from home spending before and after retirement by social security income tertiles (PSID category).}
\begin{tabularx}{\textwidth}{lCCCC}

\toprule
{SS\_Tertiles}&{First}&{Second}&{Third}&{All} \tabularnewline
\midrule\addlinespace[1.5ex]
Means:&&&& \tabularnewline
\midrule Pre-retirement&964&1,357&1,986&1,597 \tabularnewline
Post-retirement&955&1,338&2,070&1,624 \tabularnewline
Percent Change in Means&-0.9&-1.9&8.7&1.7 \tabularnewline
\midrule Medians:&&&& \tabularnewline
\midrule Pre-retirement&412&720&1,280&983 \tabularnewline
Post-retirement&466&689&1,242&932 \tabularnewline
Percent Change in Medians&13.2&-4.4&-2.9&-5.2 \tabularnewline
Median Percent Change (p10)*&-1.0&-0.9&-0.8&-0.9 \tabularnewline
Median Percent Change (p25)*&-0.8&-0.6&-0.5&-0.5 \tabularnewline
Median Percent Change (p50)&-0.2&0.0&-0.1&-0.1 \tabularnewline
Median Percent Change (p75)*&0.9&0.6&0.7&0.7 \tabularnewline
Median Percent Change (p90)*&3.2&2.6&2.5&2.6 \tabularnewline
\bottomrule \addlinespace[1.5ex]

\end{tabularx}
\begin{flushleft}
\footnotesize *These values are not medians but percentiles, as indicated in the parentheses. \linebreak --- \linebreak This table references Table 2 of Hurd and Rohwedder's paper: Heterogeneity in spending change at retirement. \linebreak --- \linebreak This spending category is defined by dining out in CAMS. \linebreak --- \linebreak Mean percent change is not reported because observation error on spending can produce large outliers when spending is put in ratio form. \linebreak --- \linebreak N = 893.
\end{flushleft}
\end{table}
