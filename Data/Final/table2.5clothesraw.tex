\begin{table}[tbp] \centering
\newcolumntype{C}{>{\centering\arraybackslash}X}

\caption{Real clothing spending before and after retirement by social security income tertiles (PSID category).}
\begin{tabularx}{\textwidth}{lCCCC}

\toprule
{SS\_Tertiles}&{First}&{Second}&{Third}&{All} \tabularnewline
\midrule\addlinespace[1.5ex]
Means:&&&& \tabularnewline
\midrule Pre-retirement&1,080&841&1,114&979 \tabularnewline
Post-retirement&520&747&1,080&872 \tabularnewline
Percent Change in Means&-51.8&-8.7&-3.1&-10.9 \tabularnewline
\midrule Medians:&&&& \tabularnewline
\midrule Pre-retirement&327&400&515&471 \tabularnewline
Post-retirement&216&300&491&392 \tabularnewline
Percent Change in Medians&-34.0&-25.0&-4.6&-16.9 \tabularnewline
Median Percent Change (p10)*&-1.0&-1.0&-0.8&-0.9 \tabularnewline
Median Percent Change (p25)*&-1.0&-0.8&-0.6&-0.7 \tabularnewline
Median Percent Change (p50)&-0.5&-0.3&-0.2&-0.2 \tabularnewline
Median Percent Change (p75)*&0.2&0.4&0.6&0.5 \tabularnewline
Median Percent Change (p90)*&2.8&1.9&1.9&1.9 \tabularnewline
\bottomrule \addlinespace[1.5ex]

\end{tabularx}
\begin{flushleft}
\footnotesize *These values are not medians but percentiles, as indicated in the parentheses. \linebreak --- \linebreak This table references Table 2 of Hurd and Rohwedder's paper: Heterogeneity in spending change at retirement. \linebreak --- \linebreak This spending category is defined by clothing in CAMS. \linebreak --- \linebreak Mean percent change is not reported because observation error on spending can produce large outliers when spending is put in ratio form. \linebreak --- \linebreak N = 882.
\end{flushleft}
\end{table}
