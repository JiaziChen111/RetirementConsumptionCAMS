\begin{table}[tbp] \centering
\newcolumntype{C}{>{\centering\arraybackslash}X}

\caption{Real clothing spending before and after retirement by social security income tertiles (PSID category).}
\begin{tabularx}{\textwidth}{lCCCC}

\toprule
{SS\_Tertiles}&{First}&{Second}&{Third}&{All} \tabularnewline
\midrule\addlinespace[1.5ex]
Means:&&&& \tabularnewline
\midrule Pre-retirement&1,099&813&1,143&979 \tabularnewline
Post-retirement&544&715&1,112&872 \tabularnewline
Percent Change in Means&-50.5&-8.9&-2.8&-10.9 \tabularnewline
\midrule Medians:&&&& \tabularnewline
\midrule Pre-retirement&327&395&520&471 \tabularnewline
Post-retirement&226&300&500&392 \tabularnewline
Percent Change in Medians&-30.9&-24.1&-3.8&-16.9 \tabularnewline
Median Percent Change (p10)*&-100.0&-96.3&-84.3&-92.3 \tabularnewline
Median Percent Change (p25)*&-94.1&-75.4&-58.1&-67.8 \tabularnewline
Median Percent Change (p50)&-45.5&-27.6&-17.6&-21.5 \tabularnewline
Median Percent Change (p75)*&20.7&44.3&68.4&47.6 \tabularnewline
Median Percent Change (p90)*&284.9&175.2&189.7&189.7 \tabularnewline
\bottomrule \addlinespace[1.5ex]

\end{tabularx}
\begin{flushleft}
\footnotesize *These values are not medians but percentiles, as indicated in the parentheses. \linebreak --- \linebreak This table references Table 2 of Hurd and Rohwedder's paper: Heterogeneity in spending change at retirement. \linebreak --- \linebreak This spending category is defined by clothing in CAMS. \linebreak --- \linebreak Mean percent change is not reported because observation error on spending can produce large outliers when spending is put in ratio form. \linebreak --- \linebreak N = 882.
\end{flushleft}
\end{table}
