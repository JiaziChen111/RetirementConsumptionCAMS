\begin{table}[tbp] \centering
\newcolumntype{C}{>{\centering\arraybackslash}X}

\caption{Real food spending before and after retirement by wealth tertiles (Generated category).}
\begin{tabularx}{\textwidth}{lCCCC}

\toprule
{Wealth\_Tertiles}&{First}&{Second}&{Third}&{All} \tabularnewline
\midrule\addlinespace[1.5ex]
Means:&&&& \tabularnewline
\midrule Pre-retirement&3,976&5,625&6,822&5,460 \tabularnewline
Post-retirement&3,870&5,477&7,109&5,501 \tabularnewline
Percent Change in Means&-2.7&-3.7&7.2&0.8 \tabularnewline
\midrule Medians:&&&& \tabularnewline
\midrule Pre-retirement&2,983&4,756&6,207&4,523 \tabularnewline
Post-retirement&3,129&4,507&5,890&4,288 \tabularnewline
Percent Change in Medians&4.9&-5.2&-5.1&-5.2 \tabularnewline
Median Percent Change (p10)*&-77.4&-66.9&-55.6&-68.7 \tabularnewline
Median Percent Change (p25)*&-47.5&-36.8&-27.6&-36.6 \tabularnewline
Median Percent Change (p50)&2.7&1.6&-3.2&-0.7 \tabularnewline
Median Percent Change (p75)*&63.8&54.0&32.6&47.3 \tabularnewline
Median Percent Change (p90)*&255.9&143.4&107.4&174.0 \tabularnewline
\bottomrule \addlinespace[1.5ex]

\end{tabularx}
\begin{flushleft}
\footnotesize *These values are not medians but percentiles, as indicated in the parentheses. \linebreak --- \linebreak This table references Table 2 of Hurd and Rohwedder's paper: Heterogeneity in spending change at retirement. \linebreak --- \linebreak This spending category is defined by food/drink and dining out in CAMS. \linebreak --- \linebreak Mean percent change is not reported because observation error on spending can produce large outliers when spending is put in ratio form. \linebreak --- \linebreak N = 919.
\end{flushleft}
\end{table}
