\begin{table}[tbp] \centering
\newcolumntype{C}{>{\centering\arraybackslash}X}

\caption{Real food spending before and after retirement by wealth tertiles (Generated category).}
\begin{tabularx}{\textwidth}{lCCCC}

\toprule
{Wealth\_Tertiles}&{First}&{Second}&{Third}&{All} \tabularnewline
\midrule\addlinespace[1.5ex]
Means:&&&& \tabularnewline
\midrule Pre-retirement&4,083&5,555&6,828&5,460 \tabularnewline
Post-retirement&3,868&5,532&7,020&5,501 \tabularnewline
Percent Change in Means&-5.3&-0.6&4.7&0.8 \tabularnewline
\midrule Medians:&&&& \tabularnewline
\midrule Pre-retirement&3,044&4,756&6,241&4,523 \tabularnewline
Post-retirement&3,108&4,655&5,766&4,288 \tabularnewline
Percent Change in Medians&2.1&-2.1&-7.6&-5.2 \tabularnewline
Median Percent Change (p10)*&-0.8&-0.7&-0.6&-0.7 \tabularnewline
Median Percent Change (p25)*&-0.5&-0.4&-0.3&-0.4 \tabularnewline
Median Percent Change (p50)&0.0&0.0&0.0&0.0 \tabularnewline
Median Percent Change (p75)*&0.6&0.5&0.3&0.5 \tabularnewline
Median Percent Change (p90)*&2.4&1.6&1.0&1.7 \tabularnewline
\bottomrule \addlinespace[1.5ex]

\end{tabularx}
\begin{flushleft}
\footnotesize *These values are not medians but percentiles, as indicated in the parentheses. \linebreak --- \linebreak This table references Table 2 of Hurd and Rohwedder's paper: Heterogeneity in spending change at retirement. \linebreak --- \linebreak This spending category is defined by food/drink and dining out in CAMS. \linebreak --- \linebreak Mean percent change is not reported because observation error on spending can produce large outliers when spending is put in ratio form. \linebreak --- \linebreak N = 919.
\end{flushleft}
\end{table}
